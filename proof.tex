\documentclass[french]{article}
\usepackage[utf8]{inputenc}
\usepackage[T1]{fontenc}
\usepackage{babel}
\usepackage{bbm}
\title{Théorème 1.A.1 Partie 2}
\author{Ernad Mehinagic}
\date{July 28th, 1914}
\usepackage{amsmath}
\usepackage{amsthm}
\begin{document}
\maketitle
\section{Théorème}
\newtheorem{theorem}{Théorème}
\begin{theorem}
  Soit a, b $\in$ $\mathbbm{R}$ tels que $a<b$. Alors il existe au plus une
  solution de l'équation (1.2.1) dans $L^1(]a,b[)$. Par ailleurs, si la fonction f
  est à variation bornée et continue à droite alors l'équation (1.2.1) possède
  une solution dans $L^1(]a,b[)$, donnée par
  \begin{align}
    u(x)=\frac{1}{\Gamma(1-\alpha)}\int\limits_{a-0}^{x}\frac{\mathrm{d}f(t)}{(x-t)^\alpha}
  \end{align}
  définie avec l'intégrale de Steljes.
\end{theorem}
\begin{proof}
  Soit f à variation bornée et continue à droite. Alors il existe deux fonctions
  $f_{1}$,\ $f_{2}$, croissantes, à variation bornée, continues à droite, telles que
  $f=f_{1}-f_{2}$ et on on déduit donc
  $\mathrm{d}f=\mathrm{d}f_{1}-\mathrm{d}f_{2}$. Ainsi, la formule (1) s'écrit :
  \begin{align*}
    u(x)=\frac{1}{\Gamma(1-\alpha)} \left\{ \int\limits_{a-0}^{x}\frac{\mathrm{d}f_{1}(t)}{(x-t)^\alpha}-\int\limits_{a-0}^{x}\frac{\mathrm{d}f_{2}(t)}{(x-t)^\alpha} \right\}
  \end{align*}
  ce qui nous donne la majoration suivante :
  \begin{equation*}
    \lvert u(t) \rvert\leq\frac{1}{\Gamma(1-\alpha)} \left\{ \int\limits_{a-0}^{x}\frac{\mathrm{d}f_{1}{(t)}}{(x-t)^\alpha}+\int\limits_{a-0}^{x}\frac{\mathrm{d}f_{2}{(t)}}{(x-t)^\alpha} \right\}
  \end{equation*}
  Ainsi, pour montrer que u $\in$ $L^1(]a,b[)$ il suffit de montrer que pour
  toute fonction $\varphi$ croissante, bornée, continue à droite sur $[a,b]$,
  et prolongée par 0 à gauche de a, on a bien
  \begin{equation*}
    x \to \int\limits_{a-0}^{x}\frac{\mathrm{d}\varphi(t)}{(x-t)^\alpha} \in L^1(]a,b[)
  \end{equation*}
  On a :
  \begin{multline}
    \int\limits_{a-0}^{b}\int\limits_{a-0}^{t}\frac{\mathrm{d}\varphi(\xi)}{(t-\xi)^\alpha}\
    \mathrm{d}t=\int\limits_{a-0}^{b}\int\limits_{\xi}^{b}\frac{\mathrm{d}t}{(t-\xi)^\alpha}\
    \mathrm{d}\varphi(\xi)=\frac{1}{1-\alpha}\int\limits_{a-0}^{b}(b-\xi)^{1-\alpha}\
    \mathrm{d}\varphi(\xi)\\
    \leq\frac{(b-a)^{1-\alpha}}{1-\alpha}\int\limits_{a-0}^{b}\mathrm{d}\varphi(\xi)\leq\frac{(b-a)^{1-\alpha}}{1-\alpha}\
    \varphi(b)<+\infty
  \end{multline}
  D'où u $\in$ $L^1(]a,b[)$.
  Enfin, on a d'après la preuve de l'unicité de la solution que la fonction
  \begin{equation*}
    (t,\xi)\to\frac{1}{(x-t)^{1-\alpha}(t-\xi)^{\alpha}} \in L^1(T_{x},\mathrm{d}t \otimes \mathrm{d}\varphi(\xi))
  \end{equation*}
  pour $x$ tel que $a<x<b$ et pour tout $\varphi$ défini comme précédemment.
  De plus, on déduit du théorème de Fubini que,
  \begin{multline}
    \frac{1}{\Gamma(\alpha)}\int\limits_{a}^{x} \left\{
      \frac{\mathrm{d}t}{(x-t)^{1-\alpha}} \frac{1}{\Gamma(1-\alpha)}
      \int\limits_{a-0}^{t}\frac{\mathrm{d}\varphi(\xi)}{(t-\xi)^\alpha} \right\}\\
    =\frac{1}{\Gamma(\alpha)\Gamma(1-\alpha)}\int\limits_{a-0}^{x}\mathrm{d}\varphi(\xi)\int\limits_{\xi}^{x}\frac{\mathrm{d}t}{(x-t)^{1-\alpha}(t-\xi)^{\alpha}}=\varphi(x)
  \end{multline}
  Enfin, on utilise (1) pour conclure.
\end{proof}
\bibliographystyle{plain}
\end{document}
